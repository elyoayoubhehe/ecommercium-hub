\documentclass[12pt,a4paper]{article}
\usepackage[utf8]{inputenc}
\usepackage{graphicx}
\usepackage{listings}
\usepackage{xcolor}
\usepackage{hyperref}
\usepackage{minted}

\definecolor{codegreen}{rgb}{0,0.6,0}
\definecolor{codegray}{rgb}{0.5,0.5,0.5}
\definecolor{codepurple}{rgb}{0.58,0,0.82}
\definecolor{backcolour}{rgb}{0.95,0.95,0.92}

\lstdefinestyle{mystyle}{
    backgroundcolor=\color{backcolour},   
    commentstyle=\color{codegreen},
    keywordstyle=\color{magenta},
    numberstyle=\tiny\color{codegray},
    stringstyle=\color{codepurple},
    basicstyle=\ttfamily\footnotesize,
    breakatwhitespace=false,         
    breaklines=true,                 
    captionpos=b,                    
    keepspaces=true,                 
    numbers=left,                    
    numbersep=5pt,                  
    showspaces=false,                
    showstringspaces=false,
    showtabs=false,                  
    tabsize=2
}

\lstset{style=mystyle}

\title{Amazon Product Scraper with React Frontend\\
\large Technical Documentation}
\author{EcommerciumHub Project}
\date{\today}

\begin{document}

\maketitle

\tableofcontents

\newpage

\section{Introduction}
This report details the implementation of an Amazon product scraping system with a React frontend. The project combines Python web scraping with a modern React interface, demonstrating the integration of different technologies for a seamless user experience.

\section{System Architecture}
\subsection{Overview}
The system consists of two main components:
\begin{itemize}
    \item Python Backend (Scraper and API)
    \item React Frontend (User Interface)
\end{itemize}

\section{Web Scraping}
\subsection{What is Web Scraping?}
Web scraping is the automated extraction of data from websites. In our case, we're extracting product information from Amazon.

\subsection{Implementation Details}
Our scraper uses several Python libraries:
\begin{lstlisting}[language=Python]
import requests
from bs4 import BeautifulSoup
from fake_useragent import UserAgent
\end{lstlisting}

Key features:
\begin{itemize}
    \item User-Agent Rotation
    \item Error Handling
    \item Anti-blocking Measures
\end{itemize}

\subsection{Scraping Process}
\begin{lstlisting}[language=Python]
def scrape_amazon(search_term):
    # Configure headers
    headers = {
        'User-Agent': UserAgent().random,
        'Accept-Language': 'en-US,en;q=0.9'
    }
    
    # Make request
    url = f"https://www.amazon.com/s?k={search_term}"
    response = requests.get(url, headers=headers)
    
    # Parse HTML
    soup = BeautifulSoup(response.text, 'html.parser')
    
    # Extract product data
    products = []
    for item in soup.find_all('div', {'data-component-type': 's-search-result'}):
        product = {
            'title': extract_title(item),
            'price': extract_price(item),
            'rating': extract_rating(item),
            # ... more fields
        }
        products.append(product)
    
    return products
\end{lstlisting}

\section{API Integration}
\subsection{What is an API?}
An API (Application Programming Interface) is a set of rules that allows different software applications to communicate with each other.

\subsection{Flask API Implementation}
\begin{lstlisting}[language=Python]
from flask import Flask, request, jsonify
from flask_cors import CORS

app = Flask(__name__)
CORS(app)

@app.route('/api/search', methods=['GET'])
def search_products():
    search_term = request.args.get('q', '')
    products = scrape_amazon(search_term)
    return jsonify(products)
\end{lstlisting}

\section{HTTP and JSON}
\subsection{HTTP Requests}
HTTP (Hypertext Transfer Protocol) is the foundation of data communication on the web.

Types of requests used in our project:
\begin{itemize}
    \item GET: Fetching product data
    \item POST: (Future implementation for tracking products)
\end{itemize}

\subsection{JSON Data Format}
JSON (JavaScript Object Notation) is our data format for communication:
\begin{lstlisting}[language=JSON]
{
  "title": "iPhone 13",
  "price": "$699",
  "rating": "4.5",
  "reviews": "2,345",
  "image_url": "https://...",
  "product_url": "https://..."
}
\end{lstlisting}

\section{React Frontend}
\subsection{Component Structure}
\begin{lstlisting}[language=JavaScript]
// ProductSearch.tsx
export default function ProductSearch() {
  const [searchTerm, setSearchTerm] = useState('');
  const [products, setProducts] = useState([]);

  const handleSearch = async () => {
    try {
      const response = await fetch(
        `http://localhost:3001/api/search?q=${searchTerm}`
      );
      const data = await response.json();
      setProducts(data);
    } catch (err) {
      setError('Failed to fetch products');
    }
  };

  return (
    // JSX for rendering the interface
  );
}
\end{lstlisting}

\section{Frontend-Backend Communication}
\subsection{Data Flow}
1. User enters search term in React interface
2. React makes HTTP request to Python backend
3. Python scrapes Amazon and returns data
4. React receives and displays the data

\subsection{Code Integration}
The connection between React and Python:
\begin{lstlisting}[language=JavaScript]
// React side (TypeScript)
interface ScrapedProduct {
  title: string;
  price: string;
  rating: string;
  reviews: string;
  image_url: string;
  product_url: string;
}

const fetchProducts = async (searchTerm: string): 
  Promise<ScrapedProduct[]> => {
  const response = await fetch(
    `http://localhost:3001/api/search?q=${searchTerm}`
  );
  return response.json();
};
\end{lstlisting}

\section{Error Handling}
\subsection{Backend Error Handling}
\begin{lstlisting}[language=Python]
@app.route('/api/search', methods=['GET'])
def search_products():
    try:
        search_term = request.args.get('q', '')
        products = scrape_amazon(search_term)
        return jsonify(products)
    except Exception as e:
        return jsonify({'error': str(e)}), 500
\end{lstlisting}

\subsection{Frontend Error Handling}
\begin{lstlisting}[language=JavaScript]
const handleSearch = async () => {
  try {
    setIsLoading(true);
    const data = await fetchProducts(searchTerm);
    setProducts(data);
  } catch (err) {
    setError('Failed to fetch products');
  } finally {
    setIsLoading(false);
  }
};
\end{lstlisting}

\section{Running the Application}
\subsection{Setup Instructions}
1. Install Python dependencies:
\begin{lstlisting}[language=bash]
pip install -r requirements.txt
\end{lstlisting}

2. Install Node.js dependencies:
\begin{lstlisting}[language=bash]
npm install
\end{lstlisting}

3. Run both servers:
\begin{lstlisting}[language=bash]
npm run dev
\end{lstlisting}

\section{Conclusion}
This project demonstrates the power of combining different technologies:
\begin{itemize}
    \item Python's strong web scraping capabilities
    \item Flask's lightweight API framework
    \item React's dynamic user interface
    \item Modern web protocols (HTTP, JSON)
\end{itemize}

The result is a seamless application that allows users to search and compare Amazon products with a modern, responsive interface.

\end{document} 