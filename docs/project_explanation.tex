\documentclass{article}
\usepackage[utf8]{inputenc}
\usepackage{graphicx}
\usepackage{listings}
\usepackage{xcolor}
\usepackage{hyperref}
\usepackage{enumitem}

\definecolor{codegreen}{rgb}{0,0.6,0}
\definecolor{codegray}{rgb}{0.5,0.5,0.5}
\definecolor{codepurple}{rgb}{0.58,0,0.82}
\definecolor{backcolour}{rgb}{0.95,0.95,0.92}

\lstdefinestyle{mystyle}{
    backgroundcolor=\color{backcolour},   
    commentstyle=\color{codegreen},
    keywordstyle=\color{magenta},
    numberstyle=\tiny\color{codegray},
    stringstyle=\color{codepurple},
    basicstyle=\ttfamily\footnotesize,
    breakatwhitespace=false,         
    breaklines=true,                 
    captionpos=b,                    
    keepspaces=true,                 
    numbers=left,                    
    numbersep=5pt,                  
    showspaces=false,                
    showstringspaces=false,
    showtabs=false,                  
    tabsize=2
}

\lstset{style=mystyle}

\title{E-commerce Project Technical Documentation\\
\large From HTML to Modern Web Development}
\author{Project Documentation Team}
\date{\today}

\begin{document}

\maketitle
\tableofcontents
\newpage

\section{Introduction}
This document explains our e-commerce project from the ground up, starting with basic concepts and moving to advanced features. It's designed for developers with HTML/CSS knowledge who want to understand modern web development.

\section{Basic Concepts}
\subsection{From HTML to Components}
In traditional HTML, you might write:
\begin{lstlisting}[language=HTML]
<div class="navbar">
    <a href="home.html">Home</a>
    <a href="products.html">Products</a>
</div>
\end{lstlisting}

In React, we use components - reusable pieces of code:
\begin{lstlisting}[language=JavaScript]
// Navbar.tsx
function Navbar() {
    return (
        <div className="navbar">
            <Link to="/home">Home</Link>
            <Link to="/products">Products</Link>
        </div>
    );
}
\end{lstlisting}

\subsection{What is a Component?}
Think of a component like a custom HTML element. Just like \texttt{<div>} or \texttt{<p>}, a component is a building block. For example, our shopping cart component (\texttt{CartSheet.tsx}) is like a pre-made shopping cart that we can use anywhere.

\section{Project Structure}
\subsection{Key Technologies}
\begin{itemize}
    \item \textbf{React}: Library for building user interfaces
    \item \textbf{TypeScript}: Adds type checking to JavaScript
    \item \textbf{Vite}: Tool that makes development faster
    \item \textbf{Tailwind CSS}: Makes styling easier with pre-made classes
\end{itemize}

\subsection{Folder Structure}
\begin{itemize}
    \item \texttt{/src}
    \begin{itemize}
        \item \texttt{/components}: Reusable parts (navbar, cart, etc.)
        \item \texttt{/pages}: Main pages of the website
        \item \texttt{/hooks}: Special React functions
    \end{itemize}
\end{itemize}

\section{Navigation and Routing}
\subsection{Traditional vs Modern Navigation}
Traditional HTML:
\begin{lstlisting}[language=HTML]
<a href="products.html">Go to Products</a>
\end{lstlisting}

Our Project (React Router):
\begin{lstlisting}[language=JavaScript]
<Link to="/products">Go to Products</Link>
\end{lstlisting}

\subsection{Available Pages}
\begin{itemize}
    \item \texttt{/}: Welcome page
    \item \texttt{/client}: Customer homepage
    \item \texttt{/client/products}: Product listing
    \item \texttt{/client/profile}: User profile
    \item \texttt{/admin}: Admin dashboard
\end{itemize}

\section{Key Features}
\subsection{Shopping Cart}
The cart is a slide-out panel that:
\begin{itemize}
    \item Shows products added
    \item Allows quantity adjustment
    \item Calculates total price
    \item Opens when clicking cart icon
\end{itemize}

\subsection{User Profile}
The profile page includes sections for:
\begin{itemize}
    \item Personal information
    \item Account settings
    \item Order history
    \item Payment methods
\end{itemize}

\section{Learning Path}
To fully understand this project, learn these in order:

\subsection{Fundamental Level}
\begin{enumerate}
    \item HTML \& CSS (You already know this!)
    \item JavaScript Basics
    \begin{itemize}
        \item Variables and functions
        \item Arrays and objects
        \item ES6 features (arrow functions, destructuring)
    \end{itemize}
\end{enumerate}

\subsection{React Basics}
\begin{enumerate}
    \item React Components
    \item JSX syntax
    \item Props and State
    \item Hooks (useState, useEffect)
\end{enumerate}

\subsection{Advanced Concepts}
\begin{enumerate}
    \item TypeScript basics
    \item React Router for navigation
    \item State management
    \item API integration
\end{enumerate}

\section{Recommended Resources}
\begin{itemize}
    \item \href{https://react.dev}{React Official Documentation}
    \item \href{https://www.typescriptlang.org/docs/}{TypeScript Documentation}
    \item \href{https://tailwindcss.com/docs}{Tailwind CSS Documentation}
    \item Online courses:
    \begin{itemize}
        \item freeCodeCamp's JavaScript course
        \item React course on Scrimba
        \item TypeScript course on Codecademy
    \end{itemize}
\end{itemize}

\section{Conclusion}
This project uses modern web development tools to create a powerful e-commerce platform. Start with JavaScript basics, then move to React, and gradually learn the advanced concepts. Remember: every expert started as a beginner!

\end{document}
